%% start of file `main.tex'.
%% Copyright 2006-2013 Xavier Danaux (xdanaux@gmail.com).
%
% This work may be distributed and/or modified under the
% conditions of the LaTeX Project Public License version 1.3c,
% available at http://www.latex-project.org/lppl/.


\documentclass[11pt,a4paper,sans]{moderncv}        % possible options include font size ('10pt', '11pt' and '12pt'), paper size ('a4paper', 'letterpaper', 'a5paper', 'legalpaper', 'executivepaper' and 'landscape') and font family ('sans' and 'roman')

% moderncv themes
\moderncvstyle{classic}                             % style options are 'casual' (default), 'classic', 'oldstyle' and 'banking'
\moderncvcolor{blue}                               % color options 'blue' (default), 'orange', 'green', 'red', 'purple', 'grey' and 'black'
%\renewcommand{\familydefault}{\sfdefault}         % to set the default font; use '\sfdefault' for the default sans serif font, '\rmdefault' for the default roman one, or any tex font name
%\nopagenumbers{}                                  % uncomment to suppress automatic page numbering for CVs longer than one page

% character encoding
\usepackage[utf8]{inputenc}                       % if you are not using xelatex ou lualatex, replace by the encoding you are using
%\usepackage{CJKutf8}                              % if you need to use CJK to typeset your resume in Chinese, Japanese or Korean

% adjust the page margins
\usepackage[scale=0.75]{geometry}
%\setlength{\hintscolumnwidth}{3cm}                % if you want to change the width of the column with the dates
%\setlength{\makecvtitlenamewidth}{10cm}           % for the 'classic' style, if you want to force the width allocated to your name and avoid line breaks. be careful though, the length is normally calculated to avoid any overlap with your personal info; use this at your own typographical risks...

% personal data
\name{Mar\'ia Florencia}{Luaiza}
\title{Estudiante Trabajo Social / Acompañante Terap\'eutico}                               % optional, remove / comment the line if not wanted
\address{Argentina}% optional, remove / comment the line if not wanted; the "postcode city" and and "country" arguments can be omitted or provided empty
\phone[mobile]{(3472)~527~151}                   % optional, remove / comment the line if not wanted
\email{mflorencia.luaiza@gmail.com}                               % optional, remove / comment the line if not wanted
\skype{skype: maria.florencia.luaiza}
\extrainfo{24 años}                 % optional, remove / comment the line if not wanted
\photo[72pt][0.4pt]{picture}                       % optional, remove / comment the line if not wanted; '64pt' is the height the picture must be resized to, 0.4pt is the thickness of the frame around it (put it to 0pt for no frame) and 'picture' is the name of the picture file
\quote{Curriculum Vitae}                                 % optional, remove / comment the line if not wanted

% to show numerical labels in the bibliography (default is to show no labels); only useful if you make citations in your resume
%\makeatletter
%\renewcommand*{\bibliographyitemlabel}{\@biblabel{\arabic{enumiv}}}
%\makeatother
%\renewcommand*{\bibliographyitemlabel}{[\arabic{enumiv}]}% CONSIDER REPLACING THE ABOVE BY THIS

% bibliography with mutiple entries
%\usepackage{multibib}
%\newcites{book,misc}{{Books},{Others}}

%----------------------------------------------------------------------------------
%            content
%----------------------------------------------------------------------------------
\begin{document}
%\begin{CJK*}{UTF8}{gbsn}                          % to typeset your resume in Chinese using CJK
%-----       resume       ---------------------------------------------------------
\makecvtitle

\section{Presentación}
Estudiante de Trabajo Social con cursado de materias finalizado. Actualmente me encuentro realizando acompañamiento socio-educativos en contextos de encierro en las unidades penitenciarias del sur de la Provincia de Santa F\'e y formo parte del proyecto cristalizado en el año 2017 `Educación en Cárceles` pertenenciente a la Secretaria de Extensi\'on y Vinculación de la Facultad de Ciencia Pol\'itica y Relaciones Internacionales. Por otra parte, estoy participando activamente como auxiliar en los proyectos de investigaci\'on `Pr\'acticas Socio-Educativas en el encierro: Entre la correcci\'on, la incapacitación y la posibilidad. Disputas, tensiones y efectos en las configuraciones subjetivas de personas privadas de su libertad en cárceles del Sur de la provincia de Santa F\'e` y `Burocracias de a pie. Los programas de acompañamiento en las politicas de infancia y juventudes` ambos pertenecientes a la Facultad de Ciencia Politica y Relaciones Internaciones de la Universiad Nacional de Rosario.
\vspace{2mm}

\section{Educación}
\cventry
{Febrero 2012}
{Licenciatura en Trabajo Social}
{Universidad Nacional de Rosario}
{Cursado Finalizado / Año 2017}
{}
{\textit{Rosario, Argentina}}
{}
\vspace{2mm}


\section{Experiencia Laboral / Acad\'emica}
\cventry{2020\\Presente}{Acompañante territorial en taller de Narrativa Creativa}{Unidad Penitenciaria 6}{Rosario, Argentina}{}
  {
    Acompañante territorial en taller de Narrativa Creativa en el marco de la Dirección provincial de inclusión socio productiva en la Unidad Penitenciaria, Programa de Educación en Cárceles y la Bemba del Sur
  }
\vspace{3mm}

\cventry{2019\\Presente}{Coordinadora del Taller Trabajo, Derechos y Ciudadania}{Unidad Penitenciaria 6}{Rosario, Argentina}{}
  {
     Coordinadora del Taller Trabajo, Derechos y Ciudadania en el marco de cursos de competencias especificas de la Universidad Nacional de Rosario
  }
\vspace{3mm}

\cventry{2019\\Presente}{Pasante en la oficina de asistencia técnica y calidad defensiva}{Servicio Público Provincial de la Defensa Penal}{Rosario, Argentina}{}
  {
  }
\vspace{3mm}

\cventry{2019}{Acompañamiento Territorial en el "taller el fútbol como herramienta para la transformación social"}{Unidad Penitenciaria 6}{Rosario, Argentina}{}
  {
    Acompañamiento Territorial en el "taller el fútbol como herramienta para la transformación social" en el marco del programa Nueva Oportunidad
  }
\vspace{3mm}

\cventry{2018\\2019}{Acompañamiento Territorial - Programa Nueva Oportunidad}{Unidad Penitenciaria 6}{Rosario, Argentina}{}
  {
    Acompañamiento a sujetos privados de su libertad, donde se aborda al fútbol no solo en términos recreativos, sino como herramienta que genera dinámicas de trabajo en equipo, fomentando la unidad para una mayor socialización e integración.
    Responsable de articular dichas actividades con la dirección de la Unidad Penitenciaria y Desarrollo Social. Gestión y asignación del presupuesto para realizar las mismas. Desarrollo y planificación de las actividades realizadas.
  }
\vspace{3mm}

\cventry{Abril 2017\\Presente}{Acompañamiento Socio-Educativo}{Unidad Penitenciaria 3 y Unidad Penitenciaria 6}{Rosario, Argentina}{}
  {
    Acompañamiento pedagógico y socio-afectivo a sujetos privados de su libertad para la finalización de sus estudios secundarios e iniciación en carreras terciarias y universitarias.
  }
\vspace{3mm}

\cventry{Agosto 2017}{ Expositora }{ VI Seminario Internacional de ‘Movimientos Sociales, Sindicatos y 2017 Educación Popular. Historia, Política y Antropología en Latinoamérica‘}{Rosario, Argentina}{}
  {}
\vspace{3mm}

\cventry{Noviembre 2017}{ Expositora }{II Jornadas de estudiantes de Trabajo Social del Litoral. `Experiencias de escritura académica`}{Santa F\'e, Argentina}{}
  {
    Presentación y Defensa del Trabajo de producción propia, `Educación en Carceles, otro escenario posible frente a la resocialización`
  }
\vspace{3mm}

\cventry{Agosto 2016\\Presente}{Relevamientos Níveles Socio-Educativos y Culturales}{Unidad Penitenciaria 6}{Rosario, Argentina}{}
  {
    Relevamiento de datos en base a entrevistas personales con los sujetos privados de su libertad, para una posterior sistematización y análisis de los mismos para poder identificar las trayectorias socio/educativas y culturales de la población.
  }
\vspace{3mm}

\cventry{Abril 2015\\Noviembre 2017}{Pr\'acticas Profesionales}{Unidad Penitenciaria 6}{Rosario, Argentina}{}
  {
    Como estudiante fui parte del Equipo de acompañamiento para la reincersión social, que tiene por fin la progresividad de la pena. Durante estos años realizamos entrevistas a los sujetos, visitas domiciliarias y la posterior elaboración de informes socio-ambientales.
  }
\vspace{3mm}
\cventry{Agosto 2016\\Abril 2017}{Acompañamiento Terap\'eutico}{Casa Particular}{Rosario, Argentina}{}
  {
    Acompañamiento Terap\'eutico a niño con diagnóstico de TDAH, con mayor enfasís en cuestiones pedagógicas y reconocimiento de autonomía y emancipación.
  }
\vspace{2mm}

\section{Otras Experiencias}
\cventry{Abril 2018\\Presente}{Miembro}{Mesa Interuniversitaria Nacional sobre Educación en contexto de encierro}{Argentina}{}
  {
    Realización de debates, discusiones y puestas en diálogo de las diversas experiencias de extensión de los programas de educación en cárceles del Mercosur.
  }

\vspace{2mm}

\cventry{Abril 2017\\Presente}{Miembro}{La Bemba del Sur}{Rosario, Argentina}{}
  {
    Miembro Activo de la Bemba del Sur, colectivo político de talleristas en contexto de encierro, que realizan prácticas culturales y educativas en las cárceles del Sur de la provincia de Santa Fé, cuyo objetivo es promover el ejercicio de los derechos culturales y educativos de las personas privadas de su libertad.
  }

\vspace{2mm}

\cventry{Marzo 2016\\Presente}{Miembro}{Programa de Extensión `Integrando 2016`}{Rosario, Argentina}{}
  {
    Programa de inserción a la educación superior para personas privadas de su libertad en cárceles del sur de la provincia de Santa Fé, que busca promover el acceso y acompañar la integración a instancias de educación superior (terciaria y universitaria) a detenidos en Unidades Penitenciarias del sur de la provincia de Santa fe, mediante un conjunto de acciones vinculadas con aspectos singulares, colectivos, sociales, y académicos que posibiliten el sostenimiento de la práctica formativa.
  }

\vspace{2mm}

\section{Cursos}
\cventry
{Abril 2017}
{Acompañamiento Terap\'eutico. Clínica y Política}
{Universidad Nacional de Rosario}
{}
{\textit{Rosario, Argentina}}
{}
\cventry
{Junio 2018}
{Capacitación Jovénes, desigualdad y violencias(s)}
{Catédra de Criminología y Control Social - Facultad de Derecho - UNR}
{}
{\textit{Rosario, Argentina}}
{}
\vspace{2mm}

\section{Conocimientos de computaci\'on}
\cvlistitem{Windows}
\cvlistitem{Microsoft Office}
\cvlistitem{Word}
\cvlistitem{Excel}
\vspace{2mm}

\section{Idiomas}
\cvitemwithcomment{Español}{Nativo}{}
\cvitemwithcomment{Ingl\'es}{Intermedio}{}

\vspace{2mm}

%\clearpage\end{CJK*}                              % if you are typesetting your resume in Chinese using CJK; the \clearpage is required for fancyhdr to work correctly with CJK, though it kills the page numbering by making \lastpage undefined
\end{document}

%% end of file `main.tex'.
\grid
\grid
\grid
\grid
\grid
